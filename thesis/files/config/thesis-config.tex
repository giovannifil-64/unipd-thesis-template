%chktex-file 1

% Load variables
\newcommand{\myUni}{Università degli Studi di Padova}
\newcommand{\myDepartment}{Dipartimento di Matematica ``Tullio Levi-Civita''}
\newcommand{\myFaculty}{Corso di Laurea in Informatica}
\newcommand{\myTitle}{Lorem ipsum dolor sit amet, consectetur adipisci elit.}
\newcommand{\myDegree}{Tesi di Laurea} % Rimossa la voce "Triennale" poichè "Tesi di Laurea" già sottintende la laurea triennale, in quanto è il titolo accademico base
\newcommand{\profTitle}{Prof.}
\newcommand{\myProf}{Cognome Nome}
\newcommand{\graduateTitle}{Laureando}
\newcommand{\myName}{Paolino Paperino}
\newcommand{\myStudentID}{1234567}
\newcommand{\myAA}{20XX-20XX}
\newcommand{\myLocation}{Padova}
\newcommand{\myTime}{Mese 20XX}
% Acronyms
\newacronym[description={\glslink{apig}{Application Programming Interface}}]
    {api}{API}{Application Program Interface}

\newacronym[description={\glslink{sdkg}{Software Development Kit}}]
    {sdk}{SDK}{Software Development Kit}

\newacronym[description={\glslink{umlg}{Unified Modeling Language}}]
    {uml}{UML}{Unified Modeling Language}

% Glossary entries
\newglossaryentry{apig} {
    name=\glslink{apig}{API}
    text=Application Program Interface,
    sort=api,
    description={In informatica con il termine \emph{API} si indica ogni insieme di procedure disponibili al programmatore, di solito raggruppate a formare un set di strumenti specifici per l'espletamento di un determinato compito all'interno di un certo programma. La finalità è ottenere un'astrazione, di solito tra l'hardware e il programmatore o tra software a basso e quello ad alto livello semplificando così il lavoro di programmazione}
}

\newglossaryentry{sdkg} {
    name=\glslink{sdkg}{SDK}
    text=SDK,
    sort=sdk,
    description={A software development kit (SDK) is a collection of software development tools in one installable package. They facilitate the creation of applications by having a compiler, debugger and sometimes a software framework. They are normally specific to a hardware platform and operating system combination. To create applications with advanced functionalities such as advertisements, push notifications, etc; most application software developers use specific software development kits}
}

\newglossaryentry{umlg} {
    name=\glslink{umlg}{UML}
    text=UML,
    sort=uml,
    description={In ingegneria del software \emph{Unified Modeling Language} (ing. linguaggio di modellazione unificato) è un linguaggio di modellazione e specifica basato sul paradigma object-oriented. L'\emph{UML} svolge un'importantissima funzione di ``lingua franca'' nella comunità della progettazione e programmazione a oggetti. Gran parte della letteratura di settore usa tale linguaggio per descrivere soluzioni analitiche e progettuali in modo sintetico e comprensibile a un vasto pubblico}
}


%\newglossaryentry{qiskitg} {
    %name=\glslink{qiskit}{Qiskit},
    %text=Qiskit,
    %sort=qiskit,
    %description={Qiskit ([quiss-kit] noun, software) %is an open-source SDK for working with quantum %computers at the level of extended quantum %circuits, operators, and primitives.}
%}

\makeglossaries

\glsaddall

\bibliography{appendix/bibliography}

\defbibheading{bibliography} {
    \cleardoublepage
    \phantomsection
    \addcontentsline{toc}{chapter}{\bibname}
    \chapter*{\bibname\markboth{\bibname}{\bibname}}
}

%Improvements the paragraph command
\titleformat{\paragraph}
{\normalfont\normalsize\bfseries}{\theparagraph}{1em}{}
\titlespacing*{\paragraph}
{0pt}{3.25ex plus 1ex minus .2ex}{1.5ex plus .2ex}

% Custom hyphenation rules
\hyphenation {
    e-sem-pio
    ex-am-ple
}

% Images path
\graphicspath{{img/}}

% Force page color, as some editors set a grayish color as default
\pagecolor{white}

%Define custom colors
% \definecolor{lavenderindigo}{rgb}{0.58, 0.34, 0.92}
\definecolor{hyperColor}{HTML}{0041EB}

\captionsetup{
    tableposition=top,
    figureposition=bottom,
    font=small,
    format=hang,
    labelfont=bf
}

\hypersetup{
    %hyperfootnotes=false,
    %pdfpagelabels,
    colorlinks=true,
    linktocpage=true,
    pdfstartpage=1,
    pdfstartview=,
    breaklinks=true,
    pdfpagemode=UseNone,
    pageanchor=true,
    pdfpagemode=UseOutlines,
    plainpages=false,
    bookmarksnumbered,
    bookmarksopen=true,
    bookmarksopenlevel=1,
    hypertexnames=true,
    pdfhighlight=/O,
    %nesting=true,
    %frenchlinks,
    %urlcolor=lavenderindigo,
    %linkcolor=blue,
    %citecolor=webgreen,
    %pagecolor=blue,
    allcolors = hyperColor,
}

% Delete all links and show them in black, useful for printing
% \hypersetup{draft}

% Listings setup
\lstset{
    language=[LaTeX]Tex,%C++,
    keywordstyle=\color{RoyalBlue}, %\bfseries,
    basicstyle=\small\ttfamily,
    %identifierstyle=\color{NavyBlue},
    commentstyle=\color{Green}\ttfamily,
    stringstyle=\rmfamily,
    numbers=none, %left,%
    numberstyle=\scriptsize, %\tiny
    stepnumber=5,
    numbersep=8pt,
    showstringspaces=false,
    breaklines=true,
    frameround=ftff,
    frame=single
}

\newcommand{\sectionname}{sezione}
\addto\captionsitalian{\renewcommand{\figurename}{Figura}
                       \renewcommand{\tablename}{Tabella}}

\newcommand{\glsfirstoccur}{\ap{{[g]}}}

\newcommand{\intro}[1]{\emph{\textsf{#1}}}

% Risks environment
\newcounter{riskcounter} % define a counter
\setcounter{riskcounter}{0} % set the counter to some initial value
%%%% Parameters
% #1: Title
\newenvironment{risk}[1]{
    \refstepcounter{riskcounter} % increment counter
    \par \noindent % start new paragraph
    \textbf{\arabic{riskcounter}. #1} % display the title before the content of the environment is displayed
}{
    \par\medskip
}

\newcommand{\riskname}{Rischio}
\newcommand{\riskdescription}[1]{\textbf{\\Descrizione:} #1.}
\newcommand{\risksolution}[1]{\textbf{\\Soluzione:} #1.}

% Use case environment
\newcounter{usecasecounter} % define a counter
\setcounter{usecasecounter}{0} % set the counter to some initial value

%%%% Parameters
% #1: ID
% #2: Nome
\newenvironment{usecase}[2]{
    \renewcommand{\theusecasecounter}{\usecasename #1}  % this is where the display of the counter is overwritten/modified
    \refstepcounter{usecasecounter} % increment counter
    \vspace{10pt}
    \par \noindent % start new paragraph
    {\large \textbf{\usecasename #1: #2}} % display the title before the content of the environment is displayed
    \medskip
}{
    \medskip
}
\newcommand{\usecasename}{UC}
\newcommand{\usecaseactors}[1]{\textbf{\\Attori Principali:} #1. \vspace{4pt}}
\newcommand{\usecasepre}[1]{\textbf{\\Precondizioni:} #1. \vspace{4pt}}
\newcommand{\usecasedesc}[1]{\textbf{\\Descrizione:} #1. \vspace{4pt}}
\newcommand{\usecasepost}[1]{\textbf{\\Postcondizioni:} #1. \vspace{4pt}}
\newcommand{\usecasealt}[1]{\textbf{\\Scenario Alternativo:} #1. \vspace{4pt}}

% Namespace description environment
\newenvironment{namespacedesc}{
    \vspace{10pt}
    \par \noindent  % start new paragraph
    \begin{description}
}{
    \end{description}
    \medskip
}

\newcommand{\classdesc}[2]{\item[\textbf{#1:}] #2}

%\linespread{1.25}
\setstretch{1.5}

%\hypersetup{pdfstartview=} commentato il 9/11/23

\newcommand{\textbfit}[1]{\textbf{\textit{#1}}}

\pagestyle{fancy}
\fancyhf{}
\fancyhead[L]{\leftmark}
% \fancyhead[R]{\thepage} %For plading the page numer on the top right
\fancyfoot[C]{\thepage} %For plading the page numer on the bottom of the page