% Acronyms
\newacronym[description={\glslink{apig}{Application Programming Interface}}]
    {api}{API}{Application Program Interface}

\newacronym[description={\glslink{sdkg}{Software Development Kit}}]
    {sdk}{SDK}{Software Development Kit}

\newacronym[description={\glslink{umlg}{Unified Modeling Language}}]
    {uml}{UML}{Unified Modeling Language}

%\newacronym[description={\glslink{qiskitg}{Quantum Information Science Toolkit}}]
    %{qiskit}{Qiskit}{Quantum Information Science Toolkit}



% Glossary entries
\newglossaryentry{apig} {
    name=\glslink{api}{API},
    text=Application Program Interface,
    sort=api,
    description={In informatica con il termine \emph{API} si indica ogni insieme di procedure disponibili al programmatore, di solito raggruppate a formare un set di strumenti specifici per l'espletamento di un determinato compito all'interno di un certo programma. La finalità è ottenere un'astrazione, di solito tra l'hardware e il programmatore o tra software a basso e quello ad alto livello semplificando così il lavoro di programmazione}
}

\newglossaryentry{sdkg} {
    name=\glslink{sdk}{SDK},
    text=SDK,
    sort=sdk,
    description={A software development kit (SDK) is a collection of software development tools in one installable package. They facilitate the creation of applications by having a compiler, debugger and sometimes a software framework. They are normally specific to a hardware platform and operating system combination. To create applications with advanced functionalities such as advertisements, push notifications, etc; most application software developers use specific software development kits}
}

\newglossaryentry{umlg} {
    name=\glslink{uml}{UML},
    text=UML,
    sort=uml,
    description={In ingegneria del software \emph{Unified Modeling Language} (ing. linguaggio di modellazione unificato) è un linguaggio di modellazione e specifica basato sul paradigma object-oriented. L'\emph{UML} svolge un'importantissima funzione di ``lingua franca'' nella comunità della progettazione e programmazione a oggetti. Gran parte della letteratura di settore usa tale linguaggio per descrivere soluzioni analitiche e progettuali in modo sintetico e comprensibile a un vasto pubblico}
}


%\newglossaryentry{qiskitg} {
    %name=\glslink{qiskit}{Qiskit},
    %text=Qiskit,
    %sort=qiskit,
    %description={Qiskit ([quiss-kit] noun, software) %is an open-source SDK for working with quantum %computers at the level of extended quantum %circuits, operators, and primitives.}
%}
